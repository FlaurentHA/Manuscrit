\documentclass{article}
\usepackage[utf8]{inputenc}
\usepackage{float}
\usepackage{amsmath}
\usepackage{systeme}
\usepackage{bbold}
\usepackage{graphicx}
\usepackage{float}
\usepackage{caption}
\usepackage{subcaption}
\usepackage{fancyhdr}
\usepackage{tcolorbox}
\usepackage{mathrsfs}
\usepackage{fancybox}
\usepackage{calc}
\usepackage{setspace}               % for LINE SPACING
\usepackage{physics}
%\numberwithin{equation}{section}

%\definecolor{main}{HTML}{5989cf}    % setting main color to be used
%\definecolor{sub}{HTML}{cde4ff}     % setting sub color to be used

%\tcbset{
%    sharp corners,
%    colback = white,
%    before skip = 0.2cm,    % add extra space before the box
%    after skip = 0.5cm      % add extra space after the box
%}                           % setting global options for tcolorbox

\newtcolorbox{boxA}{
    sharpish corners, % better drop shadow
    boxrule = 0pt,
    toprule = 4.5pt, % top rule weight
    %%enhanced,
    %%fuzzy shadow = {0pt}{-2pt}{-0.5pt}{0.5pt}{black!35} % {xshift}{yshift}{offset}{step}{options} 
}



\usepackage[T1]{fontenc}
\usepackage{geometry}
\geometry{verbose,tmargin=3cm,bmargin=3cm,lmargin=3cm,rmargin=3cm}
\usepackage{units}
\usepackage{textcomp}
\usepackage{amstext}
\usepackage[numbers]{natbib}

%\makeatletter

%%%%%%%%%%%%%%%%%%%%%%%%%%%%%% LyX specific LaTeX commands.
\newcommand{\noun}[1]{\textsc{#1}}
%% Because html converters don't know tabularnewline
\providecommand{\tabularnewline}{\\}

%%%%%%%%%%%%%%%%%%%%%%%%%%%%%% User specified LaTeX commands.
\usepackage{fancyhdr}

\pagestyle{fancy}
\fancyhf{}
\rhead{Flaurent \textsc{HEULLY--ALARY -- Internship Report}}
\lhead{\textsc{University Paul Sabatier}}
\cfoot{\thepage}

\makeatother

\usepackage{babel}
\makeatletter
\addto\extrasfrench{%
   \providecommand{\og}{\leavevmode\flqq~}%
   \providecommand{\fg}{\ifdim\lastskip>\z@\unskip\fi~\frqq}%
}

%\makeatother
\begin{document}
%\begin{center}
%\noun{\large{}University Paul Sabatier 2021}{\large\par}
%\par\end{center}

%\begin{center}
%\par\end{center}
\begin{titlepage}
\newcommand{\HRule}{\rule{\linewidth}{0.5mm}}
\center
\textsc{\LARGE
University \noun{Paul Sabatier}
} \\[1cm]
%\includegraphics[scale=0.5]{LOGO.jpg} \\[1cm]
\HRule\\[0.4cm]
{ \huge \bfseries Un joli titre \\[0.15cm] }
\HRule\\[1.5cm]
Flaurent HEULLY--ALARY
\\[1cm]
\today \\ [1cm]
\end{titlepage}
\begin{center}
\thispagestyle{plain}
\par\end{center}
\large
\tableofcontents
\newpage
\section{Introduction}

\section{Model Hamiltonian}

\subsection{Spin Hamiltonian}
For a long time theory has tried to match experiments, reproducing their results and understanding the mechanism at the origin of the different properties studies.
%Mini intro
The most well-know Spin Hamiltonian used to describe the magnetic interaction between a pair of spin located on different magnetic centres is the Heiseinberg-Dirac-Van Vleck (HDVV) Hamiltonina, written as:
\begin{equation}\label{Hheis}
    \hat{H}_{HDVV}=\sum_{i,j} J_{ij} \hat{\vb S_i} \cdot \hat{\vb S_j}
\end{equation}

where $J_{i,j}$ is the coupling constant, $\hat{\vb S_i}$ and $\hat{\vb S_j}$ are the spin operators working on site $i$ and $j$.
Several conventions exist for this Hamiltonian, with negative sign and/or a factor of 2 in front. 
As the intention of this work was to communicate with theoretical physicist we will use remove any factor in front of eq(\ref{Hheis}) unless mentioned. 
The coupling constant $J$ can be either positive or negative depending on the magnetic properties of the system, a negative value indicate ferromagnetism with magnetic moment aligned while positive indicate antiferromagnetism with opposite alignement of magnetic moments.
This Hamiltonian is regarded as a Spin Hamiltonian as it only invoke the spin degree of freedom of the system and is still to this day used in the description of the isotropic coupling in magnetic systems.

\subsection{Single centre Anisotropy}

\par Effects such as Spin-Orbit-Coupling (SOC) tend to create anisotropy in the system that cannot be described only by eq\ref{Hheis}. 
More specifically on mononuclear systems (only one magnetic centre) with a ground state of spin larger or equal to one, a lift of degeneracy between the different $M_S$ component of a same $S$ state can be observed even in the absence of magnetic field, this phenomenon is called Zero-Field-Splitting.
The associated Spin Hamiltonian is written:
\begin{equation}\label{SDS}
    \hat{H}_{ZFS}=\hat{\vb S} \cdot \overline{\overline{D}} \cdot \hat{\vb S}
\end{equation}
Where $\hat{\vb S}$ is the spin vector of the ground state and $\overline{\overline{D}}$ is a two rank symmetric tensor, in an arbitrary frame it is composed of six different parameters.
\begin{equation}
    \overline{\overline{D}}=\begin{pmatrix}
        D_{xx} & D_{xy} & D_{xz}\\
        D_{xy} & D_{yy} & D_{yz}\\
        D_{xz} & D_{yz} & D_{zz}
    \end{pmatrix}
\end{equation}
Diagonalising this tensor reduces the parameters to three which are expressed in magnetic anisotropy axes defined as the tensor principal axes. 
Working with traceless tensor allow us to use only two parameters, by convention the z-axis is taken as the main magnetic axis making D the axial parameter and E the rhombic term describing the difference in 
\begin{equation}\label{ParametreD}
    D=D_{zz}-\frac{1}{2}(D_{xx}+D_{yy})=\frac{3}{2}D_{zz}
\end{equation}
and the rhombic term:
\begin{equation}\label{ParametreE}
    E=\frac{1}{2}(D_{xx}-D_{yy})
\end{equation}
With $|D| \geq 3E \geq 0$. 
The ZFS Hamiltonian can then be written:
\begin{equation}  %Différentes conventions??%
    \hat{H}_{ZFS}=D (\hat{S}_z^2-\frac{1}{3}\hat{S}^2)+E(\hat{S}_x^2-\hat{S}_y^2)
\end{equation}
A positive value of D indicates that the ground state is mainly composed of the $M_S=0$ component meaning that the projection of the spin moment along the $z$-axis is close to zero resulting in a easy-plane magnetism. 
On the contrary if D<0, the ground state is formed from the $M_S=\pm M_{Smax}$ components with a maximum projection of spin moment along the $z$-axis resulting in an easy-axis magnetism.
For a ground state with spin larger than one and a half, other terms may appear but will not be reported in this work as the case never occured.

In the case of a triplet S=1 ground stat such as a nickel Ni (II) complex that will be studied, the matrix representation of hamiltonian\ref{SDS} in a random set of axis:
\begin{center}
\begin{tabular}{ c | c c c}
    $H_{ZFS}$ & $\ket{1,-1}$ & $\ket{1,0} $& $\ket{1,1}$ \\
    \hline
    $\ket{1,-1} $&  $\frac{1}{2}(D_{xx}-D_{yy})+D_{zz}$ & $-\frac{\sqrt{2}}{2}(D_{xz+iD_{yz}}) $ & $\frac{1}{2}(D_{xx}-D_{yy}+2iD_{xy})$ \\
    $\ket{1,0}$ & $-\frac{\sqrt{2}}{2}(D_{xz-iD_{yz}})  $& $ D_{xx}+D_{yy}  $&$ \frac{\sqrt{2}}{2}(D_{xz}+iD_{yz})$ \\
    $\ket{1,1} $& $ \frac{1}{2}(D_{xx}-D_{yy}-2iD_{xy}) $& $\frac{\sqrt{2}}{2}(D_{xz}-iD_{yz})$  & $\frac{1}{2}(D_{xx}+D_{yy})+D_{zz} $ \\
\end{tabular}\\
\end{center}
In the magnetic frame, this matrix becomes:
\begin{center}
\begin{tabular}{c | c c c}
    $H_{ZFS}$ & $\ket{1,-1}$ & $\ket{1,0} $& $\ket{1,1}$ \\
    \hline
    $\ket{1,-1}$ & $\frac{1}{2}(D_{XX}+D_{YY})+D_{ZZ}$ & 0 & $\frac{1}{2}(D_{XX}-D_{YY})$\\
    $\ket{1,0}$ & 0 & $D_{XX}+D_{YY}$ & 0\\
    $\ket{1,1}$ &  $\frac{1}{2}(D_{XX}-D_{YY})$ & 0 & $\frac{1}{2}(D_{XX}+D_{YY})+D_{ZZ}$
\end{tabular}
\end{center}
Removing the trace from the tensor and applying the convention from eq\ref{ParametreD} and eq \ref{ParametreE} we get:
\begin{center}
\begin{tabular}{c | c c c}
    $H_{ZFS}$ & $\ket{1,-1}$ & $\ket{1,0}$ & $\ket{1,1}$\\
    \hline
    $\ket{1,-1}$ & $\frac{1}{3}D$ & 0 & 0\\
    $\ket{1,0}$ & 0 & $\frac{2}{3}D $& 0\\
    $\ket{1,1}$ & 0 & 0 & $\frac{1}{3}D$
\end{tabular}
\end{center}
\subsection{Multi-centre Anisotropy}
Opening the study to multiple magnetic center introduces weak couplings between the magnetic moments of each center. 
This coupling between two sites A and B with a least of magnetic elctron each, is called exchange interaction and is described by the following Giant Spin Hamiltonian.
\begin{equation}
    \hat{H}_{AB}=\hat{\vb S}_A\cdot \overline{\overline{D}} \cdot \hat{\vb S}_B
\end{equation}
The spin-spin interaction tensor $\overline{\overline{D}}$ behaves the same way as the D tensor from Zero Field Splitting and two parameters $D$ and $E$ can be defined in the same way. 
This Hamiltonian is suitable for system with a ground state of spin larger or equal to one.
It is also possible to study cases where mixing with low-lying excited states occurs by decomposing the tensor $\overline{\overline{D}}$ in three component in the case of two unpaired electrons.

\begin{equation}
    \hat{H}_{AB}=J(\hat{\vb S}_A \cdot \hat{\vb S}_B) + \hat{\vb S}_A \cdot\overline{\overline{D}}_{AB} \cdot \hat{\vb S}_B + \vb{d}_{AB} \cdot (\hat{\vb S}_A \cross \hat{\vb S}_B)
\end{equation}

Where the first term is the isotropic exchange, extending this coupling to more than two center this term becomes equivalent to the HDVV Hamiltonian from (\ref{Hheis}).%
The two following terms describe the anisotropic exchange with the symmetric tensor of exchange $\overline{\overline{D}}_{AB}$ and the antisymmetric exchange $\vb{d}_{AB}$ also called Dzyaloshinskii-Moriya pseudo-vector interaction.
The matrix representation of two $Cu^{2+}$ ions interacting is the following:
\begin{center}
    \begin{tabular}{c | c c c c}
        $H_{MS}$ & $\ket{1,-1}$ & $\ket{1,0}$ & $\ket{1,1}$ & $\ket{0,0}$\\
        \hline
        $\ket{1,-1}$ & $\frac{J}{4}+\frac{D_{zz}}{4}$ & $\frac{D_{xz}-iD_{yz}}{2\sqrt{2}}$ & $\frac{(D_{xx}-D_{zz}-2iD_{xy})}{4} $& $\frac{d_y+id_x}{2\sqrt{2}}$\\
        $\ket{1,0}$ & $\frac{D_{xz}+iD_{yz}}{2\sqrt{2}}$ &$ \frac{J}{4} -\frac{D_{zz}}{4} +\frac{(D_{xx}+D_{yy})}{4}$& $-\frac{D_{xz}-iD_{yz}}{2\sqrt{2}}$ & -$\frac{id_z}{2}$ \\
        $\ket{1,1}$ &$\frac{(D_{xx}-D_{zz}+2iD_{xy})}{4} $ & $-\frac{D_{xz}+iD_{yz}}{2\sqrt{2}}$ & $\frac{J}{4}+\frac{D_{zz}}{4}$ & $\frac{d_y-id_x}{2\sqrt{2}}$\\
        $\ket{0,0}$ & $\frac{d_y-id_x}{2\sqrt{2}}$  & $\frac{id_z}{2}$  &$\frac{d_y+id_x}{2\sqrt{2}}$  & $-\frac{3J}{4}-\frac{D_{zz}}{4}-\frac{(D_{xx}+D_{yy})}{4}$\\
    \end{tabular}
\end{center}
From this we take that the Dzyaloshinskii-Moriya interaction create a coupling between the singlet with the three $M_S$ components of the triplet state.
As for the symmetric tensor of anisotropic exchange, it only couples the three component of the triplet which undergo a splitting in energy.
At the isotropic level, the difference between the triplet and singlet state is given by $\Delta E=J$, but with the inclusion of the anisotropic terms this becomes much more complex. 
As opposed to the Zero-Field Splitting mechanism, it is impossible to obtain values for these interactions from the energy spectrum only, as such they will be extraced from effective Hamiltonian theory.
So far, we have assumed the colinearity of spin moments, the introduction of the antisymmetric exchange now favor a \textit{spin canting} where the spin moment are no longer parallel or antiparallel resulting in weak ferromagnetism in otherwise antiferromagnetic systems.

For couplings involving more than one unpaired electron per site, local tensor are introduced:
\begin{equation}
    \hat{H}_{AB}=J(\hat{\vb S}_A \cdot \hat{\vb S}_B) + \hat{\vb S}_A \cdot\overline{\overline{D}}_{AB} \cdot \hat{\vb S}_B + \vb{d}_{AB} \cdot (\hat{\vb S}_A \cross \hat{\vb S}_B) \hat{\vb S}_A \cdot\overline{\overline{D}}_{A} \cdot \hat{\vb S}_A +\hat{\vb S}_B \cdot\overline{\overline{D}}_{B} \cdot \hat{\vb S}_B
\end{equation}
All of these considerations are only possible in the strong exchange limit where the isotropic coupling is strong compared to the anisotropic interactions.

\section{Methodology}

All type of calculation discussed here after have for purpose to solve, in some way, the Schrodinger Equation:

\begin{equation}
    \hat{H}\Psi_i=E_i\Psi_i
\end{equation}
Where $\hat{H}$ is an Hamiltonian used to describe the system studied, $E_i$ is the energy associated to the wave-function $\Psi_i$. The solutions $E_i$ of this problem are obtained pretty straightforwardly by diagonalizing the Hamiltonian except that in most cases, the $\Psi_i$ vector are not known beforehand.

The Hamiltonian discussed before is called exact electronic Hamiltonian as it encapsulate all the electronic and nuclear interaction of the system, it is written as follow in atomic units:

\begin{equation}
    \hat{H}=-\sum_{i=1}^{N}\nabla_i^2-\sum_{A=1}^{M}\frac{1}{2M_A}\nabla_A^2%
    -\sum_{i=1}^{N}\sum_{A=1}^{M}\frac{Z_A}{|r_i-R_A|}+\sum_{i=1}^{N}\sum_{j<i}^{N}\frac{1}{r_{ij}}%
    +\sum_{A=1}^{M}\sum_{B>A}^{M}\frac{Z_A Z_B}{|R_A-R_B|}
\end{equation}
Where $r_i$ is the position vector of the $i$th electron, $R_A$ the position vector of the $A$th nucleus with atomic number $Z_A$ and mass $M_A$.
This Hamiltonian can be simplified in the context of the Born-Oppenheimer approximation, the electrons are considred to adapt instantly to the movement of the nuclei, the latter's position can be fixed and taken out of the Hamiltonian.
This approximation is justified with the fact that the electrons are much lighter than the nuclei. 

\begin{equation}\label{Helec}
    \hat{H}_{elec}=-\frac{1}{2}\sum_{i=1}^{N}\nabla_i^2%
    -\sum_{i=1}^{N}\sum_{A=1}^{M}\frac{Z_A}{|r_i-R_A|}+\sum_{i=1}^{N}\sum_{j<i}^{N}\frac{1}{|r_i-r_j|}%
\end{equation}

This Hamiltonian describes the electronic problem while the nuclei contribution is set aside in a constant, having for only effect a shift in the overall energy spectrum. 
While this greatly simplify the equations, it is still not solvable analyticaly and several approximations were developped to tackle this problem. %
One common point to all the methods that were used in this work rely on the construction of molecular orbitals (\textbf{MO}) as expansions of atomic orbitals (\textbf{AO}).

\begin{equation}
    \psi_k=\sum_{i}c_i \phi_i
\end{equation}

Where $\psi_k$ is the $k$th MO built from the AO $\phi_i$ with the coefficient $c_i$. 
This expansion rely on a supposedly infinite number of AO but in real case application this is not achievable and will be restricted to a finite number of basis functions.
As the eletronic Hamiltonian does not include any information about spin, it will be included within a so called spin orbital with the introduction of two orthonormal function $\alpha(\omega)$ and $\beta(\omega)$, \textit{i.e} spin up or down function, with $\omega$ an unspecified spin variable.
From each spatial molecular orbital, two spin orbital can be created such that:
\begin{equation}
    \chi_i=\begin{cases}
    \psi_i(r)\alpha(\omega)\\
    \quad or \\
    \psi_i(r)\beta(\omega)
    \end{cases}
\end{equation}
This definition for the spin orbital is well adapted for closed shell systems where all molecular orbitals are doubly occupied, the spatial part for any spin orbital is the same for both spin function defining the restricted formalism.
Such definition changes when working with open shell system, \textit{i.e} some orbitals are singly occupied  necessiting the application of unrestricted formalism.

\subsection{Hartree-Fock Method}
The cornersone of $ab$ $initio$ calculation in quantum chemistry is the Hartree-Fock method which is usually the initial step for computing a first approximation of the wave function in molecules. 
It is a variational approach that aims to treat the N-electron problems as problem of N non interacting electrons in the presence of an average potential replicating interactions between them, it is in this sense a mean-filed theory.
This wave-function is constructed on the optimisation of a single Slater determinant $\Phi$:
\begin{equation}
    \Psi(x_1,x_2,\ldots,x_N)=\frac{1}{\sqrt{N!}}
    \begin{vmatrix}
        \chi_i (x_1) & \chi_j (x_1) & \cdots & \chi_k (x_1)\\
        \chi_i (x_2) & \chi_j (x_2) & \cdots & \chi_k (x_2)\\
        \vdots & \vdots &   &  \vdots\\
        \chi_i (x_N) & \chi_j (x_N) & \cdots & \chi_k (x_N)\\
    \end{vmatrix}
\end{equation}
Where $\chi_i$ are spin orbitals and the variable $x_i=\{r_i,\omega_i\}$, it involves all combination of all $N$ electrons in all $k$ spin orbitals. 
We introduce the shorthand notation for such determinant $\Psi(x_1,x_2,\ldots,x_N)=\ket{\chi_1\chi_2\ldots\chi_k}$ showing only the diagonal elements of the determinant.

The way to obtain the best adapted Hartree-Fock wave function comes through the resolution of the Roothan's equation:
\begin{equation}\label{Roothans}
    FC=SC\epsilon
\end{equation}
$S$ is the overlap matrix of the basis function, $\epsilon$ the matrix of orbital energies and C is the matrix of the trial vector.
The Fock operator is defined as:
\begin{equation}
    \hat{F}(i)=\hat{h}(i)+\sum_{j}^{N/2}[2\hat{J}_{j}(i)-\hat{K}_{j}(i)]
\end{equation}

The core-Hamiltonian $\hat{h}(i)$ is a mono-electronic operator that contains the kinetic energy operator of the $i$th and coulomb repulsion with the fixed nuclei. 
The two bieletronic operator $\hat{J}_{j}(i)$ and $\hat{K}_{j}(i)$ describe the coulomb repulsion and exchange mechanism between the i$th$ electron and the rest.
These operator are carry a direct dependance on the trial vectors from eq~\ref{Roothans} as such, the Roothan's equation are non-linear and will be solved iteratively following a $Self$ $Consistent$ $Field$ (SCF) procedure.


\subsection{Complete Active Space Self Consistent Field}

For most of magnetic systems, a single reference wave function is not enough as the system is usually not closed-shell and present one or more unpaired electron. Hence, its ground state is described by a wave function composed of several determinants.
One of the most used method to introduce multiple reference in the wave function is called $Complete$ $Active$ $Space$ $Self$ $Consistent$ $Field$ ($CASSCF$). 
Compared to the Hartree-Fock theory where two types of orbitals (occupied and unoccupied) were considered, in CASSCF theory orbitals are separeted in three sub-space. The inactive orbitals are doubly occupied throughout the calculation, virtual orbitals will stay empty while the active orbitals have a variable occupancy ranging from zero to two electrons.
These active orbitals define the active space into which a configuration interaction will be realised computing all the determinants possible within this sub-space. 
As the number of determinant grows significally with the number of orbitals included in the active space, this method is quite costly and its size becomes a limiting factor for the calculations.
The new wave functions is now written as an expansion of Slater determinant which is obtained via a SCF procedure where both the expansion coefficients and orbitals are optimised as to minimize the Schrodinger Equation. 
This double optimisation scheme can render the convergence troublesome, as such the definition of the active space and the choice of the starting orbitals becomes crucial. One usually starts from a set of orbitals previously obtained via an Hartree-Fock calculation.
For computation of magnetic properties in metalic systems, the minimum active space should consists of the magnetic orbitals, $i.e$ the singly occupied d-orbitals of the metalic centers, this can be extend to all of the 3-d orbitals of the metal as well as some orbitals of the ligands.
This calculation takes into account the correlation between the electrons inside the active space in the mean field created by the
other electrons. While this method provides the non-dynamical correlation, it fails to capture the correlation with the inactive electron and their reponse to excitations, called dynamic-correlation and the need to look further appears.


\subsection{Perturbation Theory}

Perturbative treatment can be applied to extract the dynamical correlation of the mono and diexcitations that is left out from simple CASSCF calculation. For such calculation, one can use the CASPT (Complete Active Space Perturbation theory)
or the NEVPT (N-electron valence state perturbation theory). Both of these methods rely on the assumption that the Hamiltonian can be partitioned into a zeroth order term and a perturbation with the parameter $\lambda$:

\begin{equation}
    \hat{H}=\hat{H}^{(0)}+\lambda\hat{H}^{(1)}+\lambda^2 \hat{H}^{(2)}+\ldots
\end{equation}


The wave function and energy are expanded in a similar way and the zeroth order term $\Psi_{(0)}^{0}$ is chosen to be the CASSCF wave function. 
The effect of the configurations outside of the active space on the energy and the wave function is estimated though perturbation theory and the expansions of the equations allows one to obtain the corrected energies.
Usually the correction are taken at second order of perturbation with CASPT2 or NEVPT2. These two method mainly differs from their chose of zeroth-order Hamiltonian, CASPT2 rely on a monoelectronic Hamiltonian built from a one electron Fock operator that falls back to the Moller-Plesset Hamiltonian in single reference case. 
This treatment may lead to what is called "intruder states" where low-lying states induce divergence in the denominator of the correction term where a difference of energy is taken. Several approach have been adapted such a the introduction of a $level$ $shift$, real or imaginary, as to fix this divergence.
In case of NEVPT2, the zeroth-order Hamiltonian is a bi-electronic Dyall Hamiltoninan which in itself incluse a shift in energy between state from inside or outside the active space preventing the appearance of intruder states.
One should add that this correlation is considered contracted as these
methods do not act upon the coefficients inside the wave functions but only on the
energies.

\subsection{Difference dedicated configurational interaction}



\subsection{Density functional theory}

Another way to get a description of the electronic structure of is through density functional theory (DFT).
The main interest of such method is the description of the ground state properties through the determination of the ground state energy $E_0$ following the variationl theorem
\begin{equation}
    E_0 = \min \bra{\Psi} \hat{H} \ket{\Psi}
\end{equation}
Here, the eletronic wavefunction of the molcule $\Psi$ is approximated to a single Slater determinant obtained from the determination of the electronic density $\rho(\vb r)$:
\begin{equation}
    \rho(\vb r)=N\int |\Psi(\vb x_1 \ldots \vb x_N)|^2 ds_1 dx_2 \ldots dx_N
\end{equation}
The Hamiltonian\ref{Helec} can be written:
\begin{equation}\label{Hop}
    \hat{H}=\hat{T} + \hat{V}_{ee} + \hat{V}_{ne}
\end{equation} 
with $\hat{T}$ the kinetic energy operator, $\hat{V}_{ee}$ the electron-electron interaction operator and $\hat{V}_{ne}$ the nuclei-electron interaction operator. 
Solving the Shrod

by replacing $v_{ne}(\rho\vb r)$ by a known external potential $v(\vb r)$, one can obtain the ground state wave function $\Psi$ by solving the Shrodinger equation giving the electron density follows.
The first Hohenberg-Kohn theorem states that this external potential is an unique functional of the electron density.
As such, knowing the electron density allows to determine the properties of the ground states.
The second Hohenberg-Kohn gives in case of non degenerate ground state, the wave function $\Psi$ is itself a functionl of $\rho(\vb)$ which allow to define the total energy:

\begin{equation}\label{EnergyDFT}
    E[\rho]= F[\rho]+ \int \rho(\vb r) v(\vb r)dr
\end{equation}

with $F[\rho]$ an universal density functional which contain the kinetic and potential contributions.
The ground state energy $E_0$ is the minimum of eq \ref{EnergyDFT} which is reached when the electron density is that of the gound state $\rho_0(\vb r)$.
In theory the knowledge of the electon densty allows to determine $E_0$, however the density dependance expression of $F[\rho]$ is not known.
\begin{equation}
    F[\rho]=T[\rho]+V_{ee}[\rho]
\end{equation}
Kohn and Sham introduced new definition of this functional by replacing the interacting system with a fictious system of N non interacting electrons the reproduces the same ground state electron density $\rho(\vb r)$.
The functional becomes:
\begin{equation}
    F[\rho]=T_s[\rho]+E_{Hxc}[\rho]
\end{equation}
Where $T_s[\rho]$ is the non interacting kinetic energy functional of density $\rho$ and $E_{Hxc}$ the Hartree-exchange-correlation functional.

\subsection{Embedded Cluster Method}

In the case of molecules, the study of magnetic anisotropy is well described by single molecule calculations as these interactions are localised between the metallic center and ligands. 
The same cannot be said about crystals, there are long range interactions that need to be taken into account. 
These systems are usually considred infinite but cannot be treated that way by the methods chosen in this work, as such we have to work on smaller part called "cluster" or "fragment".
In theory one would want the fragment to be as large as possible to reduce the error created by cutting off the fragment of its environment, unfortunately to reduce computational cost we have to limit ourselves with small sized fragments. 
It then becomes crucial to chose correctly the fragment based on our knowledge of the physics we want to include.
The properties of the cluster alone differ from the one inside its natural environment, to reproduce them accurately it is immersed inside an embedding. 
This embedding is composed of point charges and pseudo-potential that aim to replicate the electrostatic field of the crystal inside the cluster region, called Madelung Field. 
To keep the symmetries of the crystal, the point charges are positionned at the lattice site in a sphere of radius $R_c$. 

Another problem in selecting fragments in ionic crystal is the overall charge of the fragment.
The centre of the study is the metallic ions, charged positively, forming ionic bonds with negatively charged ligand.
A correct description of these magnetic centre requires to include all closest neighbors which makes the overall fragment negatively charged.
The ions at the border of the fragment, replaced by point charges, are then positive.
To avoid the electrons escaping the fragment toward them, pseudo-potentials are placed at the lattice sites near the fragment.
They act as a wall which prevent electrons from approaching these positive charges as if they were real atoms.

Three procedure were explored to create such embedding:
(1) Formal charges were used in a sphere with very large $R_c$ (around 50 \AA{}) and pseudo-potential taken from library.
(2) Optimised point charges were obtained using Ewal Summation from formal charges, this allows to reduce the $R_c$ to a few angstrom around the fragment.
(3)  


\subsection{Effective Hamiltonian}

\section{Impact of the electric field on the ZFS parameter}

\section{Electric field on Exchange anisotropy}

\section{Herbertsmithite}

\subsection{Isotropic component}

\subsection{Anisotropy}

\end{document}
